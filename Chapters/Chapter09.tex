% Chapter X

\chapter{Estensione della Base di dati alle linee ferroviarie} 
Avendo esteso il metodo sulle linee ferroviarie si deve estendere la base di dati 
progettata in precedenza. Tale operazione è molto semplice in quanto il metodo sulle stazioni e indipendente dal metodo esteso sulle linee ferroviarie.
Seguendo lo stesso modo di procedere della sezione \ref{ProgettazioneBasiDati} 
individuiamo un associazione tra le definizioni della sezione \ref{metodoLinee}
e le entità.
\begin{enumerate}
	\item railway\_routes $\leftarrow$ $\mathcal{R}$
	\item route\_segments $\leftarrow$ $\mathcal{RS}_k$
	\item route\_points $\leftarrow$ $\mathcal{RSP}_{k,s}$
	\item exposure\_route\_points $\leftarrow$ $ExpRSP$
	\item exposure\_routes $\leftarrow$ $ExpRS$
	
\end{enumerate}

\begin{table}[H]
	\centering
	\renewcommand{\arraystretch}{1.2} \small
	\begin{tabular}{|c|c|c|c|}
		\hline
		\rowcolor[HTML]{C0C0C0} 
		Entità                   & Descrizione                                                                                                                                                 & Attributi                                                           & Attributi Identificanti \\ \hline
		\multicolumn{1}{|l|}{}   & \multicolumn{1}{l|}{}                                                                                                                                       & \multicolumn{1}{l|}{}                                               & \multicolumn{1}{l|}{}   \\ \hline
		railway\_routes                  & \begin{tabular}[c]{@{}c@{}}Entità relativa alle linee ferroviarie della\\ regione di interesse\end{tabular}                                & \begin{tabular}[c]{@{}c@{}}gid\\ geom \\ name\end{tabular}                   & gid                      \\ \hline
		route\_segments                  & \begin{tabular}[c]{@{}c@{}}Entità relativa ai vari segmenti che \\ compongono le linee ferroviarie\end{tabular}                                                 & \begin{tabular}[c]{@{}c@{}}gid \\ geom\end{tabular}           & gid                     \\ \hline
		route\_points    & \begin{tabular}[c]{@{}c@{}}Entità relativa ai punti che \\ costituiscono le tratte.\end{tabular}                                     & \begin{tabular}[c]{@{}c@{}}gid \\ geom\end{tabular}     & gid                     \\ \hline
		exposure\_routes         & \begin{tabular}[c]{@{}c@{}}Entità relativa all'esposizione al rischio \\ di frana da parte di un segmento che \\ compone una linea ferroviaria \end{tabular}  & \begin{tabular}[c]{@{}c@{}}gid \\ exposure\end{tabular}          & gid                     \\ \hline
		exposure\_route\_points       & \begin{tabular}[c]{@{}c@{}}Entità relativa all'esposizione al rischio \\ frana dei punti che compongono le tratte\end{tabular}                                        & \begin{tabular}[c]{@{}c@{}}gid\\ exposure\end{tabular} & id                      \\ \hline
		
	\end{tabular}
		
	\caption{Descrizioni entità}
	\label{tab:entita}
\end{table}

\begin{table}[h] \scriptsize
	\centering
	\renewcommand{\arraystretch}{1.2}
	\begin{tabular}{|c|c|c|c|}
		\hline
		\rowcolor[HTML]{C0C0C0} 
		Associazione           & Descrizione                                                                                                                                                                                 & Entità coinvolte                                                                                                                                                    & Attributi             \\ \hline
		\multicolumn{1}{|l|}{} & \multicolumn{1}{l|}{}                                                                                                                                                                       & \multicolumn{1}{l|}{}                                                                                                                                               & \multicolumn{1}{l|}{} \\ \hline
		hanno associati           & \begin{tabular}[c]{@{}c@{}}La seguente relazione evidenzia che\\ un punto o una tratta può avere associato un valore\\ di exposure. \end{tabular} & \begin{tabular}[c]{@{}c@{}}exposure\_route\_points\textless\textgreater      route\_points\\ route\_segments \textless\textgreater exposure\_routes\end{tabular} & nessuno               \\ \hline
		Sono costituiti           & \begin{tabular}[c]{@{}c@{}}La seguente relazione evidenzia che una route \\ può essere associata con N tratte. Una tratta\\ può appartenere ad una sola route.\\ Una tratta può essere costituita da \\ n punti, un punto appartiene ad \\ una sola tratta\end{tabular} & \begin{tabular}[c]{@{}c@{}}  railway\_routes\textless\textgreater route\_segments \\ route\_points \textless\textgreater route\_segments  \end{tabular}                                                                                               & nessuno               \\ \hline
	\end{tabular}
	
	\caption{descrizioni associazioni in modo dettagliato}
	\label{tab:associazioniLinee}
\end{table}





\begin{figure}[H]
	\centering

	\begin{tikzpicture}[,node distance=2cm]
	%ExposurePoints
	\node (ExposurePoints) [process1] {exposure\_route\_\\points};
	\node (IdExposurePoint) [idEntita, below of= ExposurePoints,yshift = 1cm,xshift= 0cm] {};
	\node (Exposure) [attributo, left of = ExposurePoints,xshift = -0.5cm] {};
	
	%relazione
	\node (Relazione) [relazione,right of=ExposurePoints,xshift=2.5cm]{hanno associati};
	
	%Point	
	\node (Points) [process1, right of= Relazione, xshift= 2.5cm] {route\_points};
	\node (IdPoints) [idEntita, above of= Points,yshift = -1cm,xshift= 0cm] {};
	\node (GeomPoints) [attributo, right of = Points,xshift = 0.5cm] {};
	
	%Relazione2
	\node (Relazione2) [relazione,below of=Points,yshift=-1cm]{Sono Costituiti};
	%Tratte
	\node (Tratte) [process1, below of= Relazione2, yshift=-1cm] {route\_segments};
	\node (IdTratta) [idEntita, below of= Tratte,yshift = 1cm,xshift= -1cm] {};
	\node (geom) [attributo, right of = Tratte,xshift = 0.5cm] {};
	
	%Relazione 3
	\node (Relazione3) [relazione,left of=Tratte,xshift=-2.5cm]{Sono Costituiti};
	
	%RailwayRoutes
	\node (RailwayRoutes) [process1 , left of = Relazione3,xshift=-2.5cm] {railway\_routes};
	\node (IdRailwayRoutes) [idEntita, below of= RailwayRoutes,yshift = 1cm,xshift= -1cm] {};
	\node (geomRailwayRoutes) [attributo, below of= RailwayRoutes,yshift = 1cm,xshift= 1cm] {};
	\node (NameRailwayRoutes) [attributo, left of = RailwayRoutes,xshift = -0.5cm] {};
	
	%relazione 4
	\node (Relazione4) [relazione,below of=Tratte,yshift=-2cm]{hanno associati};
	
	%ExposureTratte
	\node (RailwayRoutesExposure) [process1 , left of = Relazione4,xshift=-2.5cm] {exposure\_routes};
	\node (IdRailwayRoutesExposure) [idEntita, below of= RailwayRoutesExposure,yshift = 1cm,xshift= -0cm] {};
	\node (ExposureRailwayRoutes) [attributo, left of = RailwayRoutesExposure,xshift = -0.5cm] {};
	
	

	%ExposurePoint
	\draw [line] (ExposurePoints.south) -| node[anchor=east,yshift = -0.5cm, xshift= -0.2cm] {gid} (IdExposurePoint.north);
	\draw [line] (ExposurePoints.west) -| node[anchor=north,yshift= 0.7cm,xshift=-0.2cm] {exposure}(Exposure.east);
	%relazione
	\draw [line] (Relazione.west) -| node[anchor=west,yshift = 0.3cm, xshift= 0.2cm] {(1,1)} (ExposurePoints);
	\draw [line] (Relazione.east) -| node[anchor=east,yshift = 0.3cm, xshift= -0.2cm] {(0,1)} (Points.west);

	%Route Points
	\draw [line] (Points.north) -| node[anchor=east,yshift = 0.5cm, xshift= -0.2cm] {gid} (IdPoints.south);
	\draw [line] (Points.east) -| node[anchor=south,yshift= -0.7cm,xshift= 0.3cm] {geom}(GeomPoints.west);
	
	%relazione2
	\draw [line] (Points.south) -| node[anchor=west,yshift = -0.5cm, xshift= 0.1cm] {(1,1)} (Relazione2);
	\draw [line] (Relazione2.south) -| node[anchor=east,yshift = -0.5cm,xshift=0.1cm]{(1,n)}(Tratte.north) ;
	
	%Tratte
	\draw [line] (Tratte.south) -| node[anchor=east,yshift = -0.5cm, xshift= -0.2cm] {gid} (IdTratta.north);
	\draw [line] (Tratte.east) -| node[anchor=north,yshift= 0.7cm,xshift=0.2cm] {geom}(geom.west);
	
	%relazione3
	\draw [line] (Tratte.west) -| node[anchor=north,yshift = 0.5cm, xshift= 0.6cm] {(1,1)} (Relazione3.east);
	\draw [line] (Relazione3.west) -| node[anchor=north,yshift = 0.5cm, xshift= 0.6cm] {(1,n)} (RailwayRoutes.east);
	%railwayRoutes
	\draw [line] (RailwayRoutes.south) -| node[anchor=east,yshift = -0.5cm, xshift= -0.2cm] {gid} (IdRailwayRoutes.north);
	\draw [line] (RailwayRoutes.south) -| node[anchor=east,yshift = -0.5cm] {geom}(geomRailwayRoutes.north);
	\draw [line] (RailwayRoutes.west) -| node[anchor=north,yshift= 0.7cm,xshift=-0.2cm] {name}(NameRailwayRoutes.east);
	
	%relazione4
	\draw [line] (Relazione4.west) -| node[anchor=north,yshift = 0.5cm, xshift= 0.6cm] {(1,1)} (RailwayRoutesExposure.east);
	\draw [line] (Tratte.south) -| node[anchor=north,yshift = -0.5cm, xshift= 0.4cm] {(0,1)} (Relazione4.north);

	
	%exposureRoutes
	\draw [line] (RailwayRoutesExposure.south) -| node[anchor=east,yshift = -0.5cm, xshift= -0.2cm] {gid} (IdRailwayRoutesExposure.north);
	\draw [line] (RailwayRoutesExposure.west) -| node[anchor=north,yshift= 0.7cm,xshift=-0.5cm] {exposure}(ExposureRailwayRoutes.east);
	
	
	\end{tikzpicture}
	\caption{rappresentazione diagramma Er della base di dati estesa. sono presenti 4 associazioni ed 5 entità.} 
	\label{fig:diagrammaEREsteso}
	
	
\end{figure}
	\begin{enumerate}
	\item railway\_routes(\underline{gid},geom,name)
	\item route\_segments(\underline{gid},id\_route,geom)
	\begin{enumerate}
		\item FK: id\_route REFERENCES railway\_routes(gid)  
	\end{enumerate} 
	\item route\_points(\underline{gid},id\_segment,geom)
	\begin{enumerate}
		\item FK: id\_segment REFERENCES route\_segments(gid)  
	\end{enumerate}
	\item exposure\_routes(\underline{gid},id\_segment,exposure)
	\begin{enumerate}
		\item FK: id\_segment REFERENCES route\_segments(gid)
	\end{enumerate}
	\item exposure\_route\_points(\underline{gid},id\_point,exposure)
	\begin{enumerate}
		\item FK: id\_point REFERENCES route\_points(gid)
	\end{enumerate}
\end{enumerate}

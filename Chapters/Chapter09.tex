% Chapter X

\chapter{Estensione della Base di dati alle linee ferroviarie} 
Avendo esteso il metodo sulle linee ferroviarie si deve estendere la base di dati 
progettata in precedenza. Tale operazione è molto semplice in quanto il metodo sulle stazioni e indipendente dal metodo esteso sulle linee ferroviarie.
Seguendo lo stesso modo di procedere della sezione \ref{ProgettazioneBasiDati} 
individuiamo un associazione tra le definizioni della sezione \ref{metodoLinee}
e le entità.
\begin{enumerate}
	\item railway\_routes $\leftarrow$ $\mathcal{R}$
	\item railway\_routes\_segment $\leftarrow$ $\mathcal{RS}_k$
	\item railway\_routes\_points $\leftarrow$ $\mathcal{RSP}_{k,s}$
	\item exposure\_segments\_points $\leftarrow$ $ExpRSP$
	\item exposure\_route\_segment $\leftarrow$ $ExpRS$
	
\end{enumerate}

\begin{figure}[H]
	\centering
	\begin{tikzpicture}[node distance=2cm]
	
	%GeoArea
	\node (GeoArea) [process1, yshift = 1cm] {GeoArea};
	\node (IdGeoArea) [idEntita, below of= GeoArea,yshift = 1cm,xshift= -1cm] {};
	\node (GeomGeoArea) [attributo, below of= GeoArea,yshift = 1cm,xshift= 1cm] {};
	
	%Zones
	\node (Zones) [process1, right of= GeoArea, xshift= 3cm] {Zones};
	\node (IdZones) [idEntita, below of= Zones,yshift = 1cm,xshift= -1cm] {};
	\node (SZK) [attributo, below of= Zones,yshift = 1cm,xshift= 1cm] {};
	\node (GeomZones) [attributo, right of = Zones,xshift = 0.5cm] {};
	
	%Isoipse
	\node (Isoipse) [process1, right of= Zones, xshift= 3cm] {Isoipse};
	\node (IdIsoipse) [idEntita, below of= Isoipse,yshift = 1cm,xshift= -1cm] {};
	\node (Elevation) [attributo, below of= Isoipse,yshift = 1cm,xshift= 1cm] {};
	\node (GeomIsoipse) [attributo, right of = Isoipse,xshift = 0.5cm] {};
	
	%RaylwayStation
	\node (RailwayStation) [process1, below of= GeoArea,yshift= -2cm] {railway\_station};
	\node (IdRailway) [idEntita, below of= RailwayStation,yshift = 1cm,xshift= -1cm] {};
	\node (NameRailwayStation) [attributo, below of= RailwayStation,yshift = 1cm,xshift= 1cm] {};
	\node (GeomRailway) [attributo, left of = RailwayStation,xshift = -0.5cm] {};
	
	%relazione
	\node (Relazione) [relazione,right of=RailwayStation,xshift=3cm]{Ha associato};
	
	%station_exposure	
	\node (StationExposure) [process1, right of= Relazione, xshift= 3cm] {station\_exposure};
	\node (IdStationExposure) [idEntita, below of= StationExposure,yshift = 1cm,xshift= -1cm] {};
	\node (StationID) [attributo, below of= StationExposure,yshift = 1cm,xshift= 1cm] {};
	\node (Exposure) [attributo, right of = StationExposure,xshift = 0.5cm] {};
	
	
	
	%GeoArea
	\draw [line] (GeoArea.south) -| node[anchor=east,yshift = -0.5cm, xshift= -0.2cm] {id} (IdGeoArea.north);
	\draw [line] (GeoArea.south) -| node[anchor=east,yshift = -0.6cm,xshift= -0.2cm] {geom}(GeomGeoArea.north);
	%Zones
	\draw [line] (Zones.south) -| node[anchor=east,yshift = -0.5cm, xshift= -0.2cm] {gid} (IdZones.north);
	\draw [line] (Zones.south) -| node[anchor=east,yshift = -0.5cm,xshift= -0.2cm] {Szk}(SZK.north);
	\draw [line] (Zones.east) -| node[anchor=south,yshift= -0.8cm] {geom}(GeomZones.west);
	%Isoipse
	\draw [line] (Isoipse.south) -| node[anchor=east,yshift = -0.5cm, xshift= -0.2cm] {gid} (IdIsoipse.north);
	\draw [line] (Isoipse.south) -| node[anchor=west,yshift = -0.5cm,xshift= 0.3cm] {Elevation}(Elevation.north);
	\draw [line] (Isoipse.east) -| node[anchor=north,yshift= 0.7cm] {geom}(GeomIsoipse.west);
	%RaylwayStation
	\draw [line] (RailwayStation.south) -| node[anchor=east,yshift = -0.5cm, xshift= -0.2cm] {gid} (IdRailway.north);
	\draw [line] (RailwayStation.south) -| node[anchor=east,yshift = -0.5cm,xshift= -0.2cm] {Name}(NameRailwayStation.north);
	\draw [line] (RailwayStation.west) -| node[anchor=north,yshift= 0.7cm] {geom}(GeomRailway.east);
	%relazione
	\draw [line] (RailwayStation.east) -| node[anchor=east,yshift = 0.3cm, xshift= -0.9cm] {(0,1)} (Relazione.west);
	\draw [line] (Relazione.west) -| node[anchor=west,yshift = 0.3cm, xshift= 1cm] {(1,1)} (RailwayStation);
	\draw [line] (Relazione.east) -| node[anchor=east,yshift = 0.3cm, xshift= -1cm] {(0,1)} (StationExposure.west);
	\draw [line] (StationExposure.west) -| node[anchor=west,yshift = 0.3cm, xshift= 0.9cm] {(1,1)} (Relazione.east);
	%station_exposure
	\draw [line] (StationExposure.south) -| node[anchor=east,yshift = -0.5cm, xshift= -0.2cm] {gid} (IdStationExposure.north);
	\draw [line] (StationExposure.south) -| node[anchor=west,yshift = -0.5cm,xshift= 0.3cm] {station\_id}(StationID.north);
	\draw [line] (StationExposure.east) -| node[anchor=north,yshift= 0.7cm,xshift= 0.3cm] {exposure}(Exposure.west);
	
	
	\end{tikzpicture}
	\caption{rappresentazione diagramma Er della base di dati. sono presenti 1 associazioni ed 5 entità.} 
	\label{fig:diagrammaEREsteso}
\end{figure}

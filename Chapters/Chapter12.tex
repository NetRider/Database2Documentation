
\chapter{Conclusioni} 
L'obiettivo di questo studio è stato quello di fornire uno strumento che permettesse, ad un organo competente, di prendere decisioni mirate riguardo la prevenzione del territorio.
Essendo il numero di asset strategici molto grande, tali decisioni permettono di indirizzare in maniera efficace le risorse disponibili alle aree a rischio maggiore. Nello studio qui proposto si è formalizzato inizialmente un metodo che permettesse di calcolare l'esposizione al rischio di frana di un punto generico all'interno di un'area geografica. Tale punto rappresenta asset strategici come scuole, case, pali della luce, ect. Il metodo è stato poi esteso a delle linee rappresentanti tutte quelle opere artificiali come strade, ferrovie, ponti, etc presenti sul territorio. A partire dal metodo è stato proposto un algoritmo che è stato implementato attraverso DBMS PostgreSQL avvalendosi dell’estensione spaziale PostGIS. Sono state implementate delle UDF e definite apposite query che permettessero di fare un'analisi critica sui risultati. Per validare e testare l'algoritmo sono stati usate, come dati di input, le stazioni e le linee ferroviarie abruzzesi. Attraverso le matrici di contingenza si sono stabilite metriche ben precise per valutare l'effettiva capacità dell'algoritmo nel riuscire a classificare in modo corretto i dati di input relativamente all'esposizione al rischio frana. I risultati ottenuti hanno dimostrato che l'algoritmo si comporta in modo buono per le stazioni migliorando ulteriormente nei casi delle linee ferroviarie. E' opportuno sottolineare come l'algoritmo non abbia la pretesa di restituire valori assoluti di esposizione al rischio, ma fornire una classifica in modo da poter fare un'analisi comparativa tra le stazioni e le linee in modo tale da individuare i casi più pericolosi. In definitiva attraverso gli spunti proposti nel capitolo \ref{ch:sviluppiFuturi} si potrebbero ulteriormente migliorare i risultati ottenuti dal metodo andando a risolvere le problematiche analizzate in dettaglio nel capitolo \ref{ch:validazione_stazioni}.

Concludendo ci auguriamo che il metodo proposto possa essere preso in esame dalle autorità competenti alla gestione della prevenzione sul territorio.